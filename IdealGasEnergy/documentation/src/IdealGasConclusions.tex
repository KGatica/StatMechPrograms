%keywords section definiton
\def\keywords{\vspace{.5em}
{\textit{Keywords}:\,\relax%
}}
\def\endkeywords{\par}

\def\abbreviations{\vspace{.5em}
{\textit{Abbreviations}:\,\relax%
}}
\def\endabbreviations{\par}

\title{Ideal Gas}

\author{
       Computational Biology 2018
}
%\date{}
\documentclass[10pt]{article}
\usepackage{amssymb} %maths
\usepackage{amsmath} %maths
\usepackage{setspace}
\usepackage{bm}
\begin{document}
\maketitle
\pagebreak
\section{Conclusions \cite{IMSSS}}

\subsection{Question 4}
From our simulation experiment, it follows that the partition function has the form:


\begin{equation} % requires amsmath; align* for no eq. number
   q=\sum_{\bf{n}}\exp[-\beta E_{\bf{n}}]=\sum_{\bf{n}}\exp\left[ -\frac{\beta h^2(n_x^2+n_y^2+n_z^2)}{8mL^2}  \right]=\left(\sum_{n}^{\infty}\exp\left[ -\frac{\beta h^2n^2}{8mL^2}  \right]\right)^3
   \label{equation:1.1} 
\end{equation}
%
where the power of 3 is due to the fact the the summations over the three quantum numbers $n_x,n_y,n_z$ are independent of each other. For the average energy energy of one particle we find:

\begin{equation} % requires amsmath; align* for no eq. number
   \langle E \rangle = \frac{1}{q}\sum_{\bf{n}}E_{\bf{n}}\exp[-\beta E_{\bf{n}}]=\frac{3}{q^{1/3}}\sum_{n=1}^{\infty}\frac{h^2n^2}{8mL^2}\exp\left[ - \frac{\beta h^2n^2}{8mL^2}  \right]
   \label{equation:1.2} 
\end{equation}
%
In the lectures we derived an aproximation for $q$ via the semiclassical approximation. We will take a similar route.

For all practical circumstances, \emph{i.e.} realistic size and temperature, many terms will contribute to the summations of eqs. \eqref{equation:1.1} and \eqref{equation:1.2}, thus they can be treated as continuous variables. Indeed for $\alpha < < 1$ we can employ:

\begin{equation} % requires amsmath; align* for no eq. number
   \sum_{n=1}^{\infty} \exp[-\alpha n^2]\approx \int_0^\infty dn\exp[-\alpha n^2]=\frac{1}{2}\sqrt{\frac{\pi}{\alpha}}
   \label{equation:1.3} 
\end{equation}
%
and

\begin{equation} % requires amsmath; align* for no eq. number
   \sum_{n=1}^{\infty} n^2\exp[-\alpha n^2]\approx \int_0^\infty dn n^2\exp[-\alpha n^2]=\frac{1}{4\alpha}\sqrt{\frac{\pi}{\alpha}}
   \label{equation:1.4} 
\end{equation}
%
Employing these approximations for eqs. \eqref{equation:1.1} and \eqref{equation:1.2}, we get:

\begin{equation} % requires amsmath; align* for no eq. number
   q=\left( \frac{\pi}{4\alpha}\right)^{3/2}=\left(  \frac{L}{\Lambda} \right)^3=\frac{V}{\Lambda^3}
   \label{equation:1.5} 
\end{equation}
%
with

\begin{equation} % requires amsmath; align* for no eq. number
   \Lambda \equiv \frac{h}{\sqrt{2\pi mk_BT}}=\frac{h}{\sqrt{2\pi m/\beta}}
   \label{equation:1.6} 
\end{equation}
%
if often called the thermal wavelength. For $\langle E\rangle $ we get 

\begin{equation} % requires amsmath; align* for no eq. number
   \langle E\rangle=\frac{3}{2\beta}=\frac{3}{2}k_BT
   \label{equation:1.7} 
\end{equation}

Try to derive the relations by yourself!!!

\bibliographystyle{unsrt}
\bibliography{ref.bib}


\end{document}