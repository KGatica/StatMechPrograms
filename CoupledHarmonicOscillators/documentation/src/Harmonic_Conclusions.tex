%keywords section definiton
\def\keywords{\vspace{.5em}
{\textit{Keywords}:\,\relax%
}}
\def\endkeywords{\par}

\def\abbreviations{\vspace{.5em}
{\textit{Abbreviations}:\,\relax%
}}
\def\endabbreviations{\par}

\title{Coupled Harmonic Oscillators}

\author{
       Computational Biology 2018      
}
%\date{}
\documentclass[10pt]{article}
\usepackage{amssymb} %maths
\usepackage{amsmath} %maths
\usepackage{setspace}
\begin{document}
\maketitle
\pagebreak
\section{Conclusions \cite{IMSSS}}

\subsection{Question 3}
From our simulation experiment in question 3, it follows that the probability distribution for
a given state $k$ of a single harmonic oscillator reads:

\begin{equation} % requires amsmath; align* for no eq. number
   P(k)=\frac{\exp[-\beta E_k]}{q}
   \label{equation:1.1} 
\end{equation}
%
with constants $\beta$ and $q$. In statistical thermodynamics the numerator $\exp[-\beta E_k]$ is called the Boltzmann factor. The constant $q$ is called the molecular partition function or molecular partition sum, defined as the sum of Boltzmann factors over all states of the oscillator.

\begin{equation} % requires amsmath; align* for no eq. number
   q=\sum_{k=0}^{\infty}\exp[-\beta E_k]
   \label{equation:1.2} 
\end{equation}
%
Substituting $x=\exp(-\beta\hbar\omega)$ we can rewrite the partition sum of $q$ of a harmonic oscillator as a geometric series employing:

\begin{equation} % requires amsmath; align* for no eq. number
   \sum_{i=0}^{\infty} x^i=\frac{1}{1-x} \quad \text{for}\quad|x|<1
   \label{equation:1.3} 
\end{equation}
%
thus:

\begin{equation} % requires amsmath; align* for no eq. number
   q=\frac{1}{1-\exp[-\beta E_k]} 
   \label{equation:1.4} 
\end{equation}

The average energy $\langle E_i\rangle$ of a single oscillator can be related to $\beta$  using the same substitution as in eq. \eqref{equation:1.3} and employing :

\begin{equation} % requires amsmath; align* for no eq. number
   \sum_{i=0}^{\infty} ix^i=\frac{x}{(1-x)^2} \quad \text{for}\quad|x|<1
   \label{equation:1.5} 
\end{equation}
%
which is eq \eqref{equation:1.3} diferentiated with respect to $x$ and multiplied by $x$, leads to the following relation:

\begin{equation} % requires amsmath; align* for no eq. number
   \langle E_i\rangle=\sum_{k=0}^{\infty}\hbar\omega kP(k)=\hbar\omega\frac{\sum_{k=0}^{\infty}k\exp[-\beta \hbar\omega k]}{\sum_{k=0}^{\infty}\exp[-\beta \hbar\omega k]}=\frac{\hbar\omega}{\exp[-\beta \hbar\omega ]-1}
   \label{equation:1.6} 
\end{equation}
%

The latter means that for big enough number of oscillators, a single oscillator $i$ is surrounded by a "heath bath". The temperature follows from the total energy E, which in turn determines  $ \langle E_i\rangle$ and $\beta$.Thus, from the average energy $ \langle E_i\rangle$ and the relation $\beta=1/k_BT$ we can estimate the temperature of the simulated system. 

\begin{enumerate}
\item For a number of system energies ($E_t$) use simulation data to find $ \langle E_i\rangle$ and $\beta$. Check your results with eq. \eqref{equation:1.6} 
\end{enumerate}


\subsection{Question 4}
From our simulation experiment in question 4, it follows that the entropy increase in time ands  reaches maximum value when we start from the less likely distribution.


\bibliographystyle{unsrt}
\bibliography{ref.bib}


\end{document}